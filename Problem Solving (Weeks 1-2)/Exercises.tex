\documentclass[11pt,letterpaper]{article}
\usepackage[utf8]{inputenc}

\usepackage[margin=1in]{geometry} 
\usepackage{preamble}
\usepackage{tocloft}
\renewcommand{\cftsecleader}{\cftdotfill{\cftdotsep}}
\begin{document}
\tableofcontents
\newpage
\lhead{EFM}
\chead{REU Exercises}
\rhead{\today}%\today
\tikzset{every picture/.style={line width=0.75pt}} %set default line width to 0.75pt        

\inter{Notation}
Generically, we will write:
\begin{itemize}
    \item  $d= \# \text{ of down cusps}$
    \item  $u= \# \text{ of up cusps}$
    \item  $r= \# \text{ of right (or left) cusps}$
    \item  $p= \# \text{ of positive crossings}$
    \item  $n= \# \text{ of negative crossings}$
\end{itemize}

\addcontentsline{toc}{section}{June 5th}
\section*{\underline{June 5th}}

\prob{1}
\begin{enumerate}[label=(\alph*)]
    \item We may write $y(t)=-3t/2$ which agrees with $z'(t)/x'(t)$ whenever the latter expression is defined.
    \item \TODO
\end{enumerate}

\prob{2}
See June\_5\_Q2.pdf.

\prob{3}

\prob{4}

We see Legendrian reiedemeister move $R1$ taking $F_1$ to $F_2$, we have
    \[\tb(F_1)=(p-n)-r \text{ and } \tb(F_2)=((p+1)-n)-(r+1)=(p-n)-r=\tb(F_1)\]
no matter the orientation. Additionally,
\[\rot(F_1)=\frac{1}{2}(d-u) \text{ and } \frac{1}{2}((d+1)-(u+1))\]


\inter{Thurston-Bennequin}

\begin{definition}
    Given a link $(L,L')$, the \textit{linking number} $\lk(L,L')$ is the signed sum of crossing numbers divided by 2.
\end{definition}

\begin{prop}
    We have $\tb(\L)=\lk(\L,\L')$ where $\L'$ is a copy of $\L$ obtained by shifting slightly in the $z$ direciton.
\end{prop}

\begin{proof}
    For every cusp and crossing in $\L$, we have 1 new crossing and 2 new crossings, respectively, in the link diagram of $(\L,\L')$. For a right cusp, we get a $+1$. For a left cusp, we get a $-1$. For a positive crossing, we get a $+2$

    \TODO
\end{proof}

\prob{5}

We can compute $\tb(\L)$ by noticing that for every right up-cusp in $\Fr(\L)$, we get a negative crossing in $\Pi(\L)$ and for every 

\prob{6}

\prob{7}
$S_+$ adds to down cusps one of which is a right cusp. $S_-$ add two up cusps one of which is a right cusp. The number of crossings stay the same so that
    \[\tb(S_\pm(\L)))=(p-n)-(r+1)=\tb(\L)-1\]
and
    \[\rot(S_+(\L))=\frac{1}{2}((d+2)-u)=\rot(\L)+1,\;\; \rot(S_-(\L))=\frac{1}{2}(d-(u+2))=\rot(\L)-1.\]

\addcontentsline{toc}{section}{June 6th}
\section*{\underline{June 6th}}

\inter{Diagram}
The following commutes
\[\begin{tikzcd}[ampersand replacement=\&]
	{\Omega^0(U)} \& {\Omega^1(U)} \& {\Omega^2(U)} \& {\Omega^3(U)} \\
	{C^\infty(U)} \& {\mathfrak{X}(U)} \& {\mathfrak{X}(U)} \& {C^\infty(U)}
	\arrow["d", from=1-1, to=1-2]
	\arrow["d", from=1-2, to=1-3]
	\arrow["d", from=1-3, to=1-4]
	\arrow["\id", from=2-1, to=1-1]
	\arrow["{\Phi_1}", from=2-2, to=1-2]
	\arrow["{\Phi_2}", from=2-3, to=1-3]
	\arrow["{\Phi_3}"', from=2-4, to=1-4]
	\arrow["\nabla", from=2-1, to=2-2]
	\arrow["{\nabla \times}", from=2-2, to=2-3]
	\arrow["\nabla\cdot", from=2-3, to=2-4]
\end{tikzcd}\]
where 
\begin{multline*}
    \Phi_1(E_1,E_2,E_3)=E_1dx+E_2dy+E_3dz, \;\; \Phi_2(E_1,E_2,E_3)= E_1 dy\wedge dz+E_2 dz\wedge dx + E_3 dz \wedge dy, \\ \text{ and } \Phi_3(h)= h dx \wedge dy \wedge dz.
\end{multline*}

\prob{1}
\TODO
\prob{2}
First prove that $d(f\wedge \lambda)=df\wedge \lambda$ for $f\in \Omega^0(U)$ and $\lambda \in \Omega^k(U)$. Next note that we only need to worry about the multiplication of one forms since $d(\lambda\wedge \eta)=0$.

Also see See June\_6\_Q2.pdf.
\prob{3}
This follows from $\nabla\times \circ \nabla=0$ and $\nabla \cdot 
 \circ \nabla \times =0$.
\prob{4}
Assuming $U$ is simply connected then we know that any vector field with trivial curl, then it is the gradient of a scalar field. So anytwo differ by gradient of scalar field have same curl by linearity.


\prob{5}
We have $\int_C \Vec{F} \cdot d\Vec{s}= \int_a^b \Vec{F}(\gamma(t))\cdot \Dot{\gamma}(t) dt$ where $\gamma$ is some parametrization of $C$. Define $f(x,y,z)=xyz$ and note that $\nabla f= \Vec{F}$. Then $\frac{d}{dt} (f\circ \gamma)= (\Vec{F} \circ \gamma)\cdot \Dot{\gamma}$ by chain rule. Hence, by FTC, we have
    \[\int_C \Vec{F} \cdot d\Vec{s} = f(\gamma(b))-f(\gamma(a))=f(1,1,1)-f(0,0,0)\]
and the left is independent of $\gamma$, as required.


\prob{6}
We claim that the following diagram commutes:
\[\begin{tikzcd}[ampersand replacement=\&]
	{\Omega^1(U)\times \Omega^1(U)} \& {\Omega^1(U)} \\
	{\mathfrak{X}(U)\times \mathfrak{X}(U)} \& {\mathfrak{X}(U)}
	\arrow["{-\wedge-}", from=1-1, to=1-2]
	\arrow["{\Phi_1\times \Phi_1}"', from=2-1, to=1-1]
	\arrow["{[-,-]}", from=2-1, to=2-2]
	\arrow["{\Phi_2}"', from=2-2, to=1-2]
\end{tikzcd}\]
Consider $\Vec{E} = \sum_i E_i \Hat{\textbf{x}}_i$ and $\Vec{F}= \sum_j F_j \Hat{\textbf{x}}_j$ and note that
    \begin{multline*}
        \left(\sum_{i} E_i dx^i\right) \wedge \left(\sum_{j} F_j dx^j\right)=\sum_{i,j} E_i F_j dx^i \wedge dx^j= \sum_{i<j} (E_i F_j - E_j F_i )dx^i \wedge dx^j\\ =(E_2 F_3-E_3 F_2)dy \wedge dz + (E_3 F_1 - E_1 F_3) dz\wedge dz + (E_1 F_2 - E_2 F_1) dx\wedge dy.
    \end{multline*}
Since $\Vec{E}\times \Vec{F} = (E_2 F_3 - E_3 F_2)\Hat{\textbf{x}}+ (E_3 F_1 - E_1 F_3) \Hat{\textbf{y}} + (E_1 F_2 - E_2 F_1) \Hat{\textbf{z}}$, the result follows.

\prob{7}
Note that \[\xi \wedge \xi =\Phi_2( \Phi_2^{-1}(\xi \wedge \xi )) = \Phi_2(\Phi_1^{-1}(\xi)\times \Phi_1^{-1}(\xi))=\Phi_2(\Vec{0})=0\]
since $\Vec{v}\times \Vec{v}=\Vec{0}$ for all $\Vec{v}\in \R^3$.

\prob{8}
\begin{enumerate}[label=(\alph*)]
    \item Note that $\alpha(\Dot{\gamma}(t))=\Dot{z}(t)-y(t) \Dot{x}(t)$, by definition.
    \item We have that $d\alpha= - dy\wedge dx = dx\wedge dy$ and hence
        \[\alpha\wedge d\alpha = dz\wedge dx 
        \wedge dy - 0 = dx\wedge dy \wedge dz.\]
\end{enumerate}

\prob{9}
This surface is a sphere of radius 1. Then we can see that
    \[\int_{\mathcal{S}}\nu = \int_{\partial \mathcal{B}}\nu=\int_{\mathcal{B}}d\nu=3\int_{\mathcal{B}}dx\wedge dy \wedge dz=4\pi\]
from Stokes' theorem.
\prob{10}
See above.

\addcontentsline{toc}{section}{June 7th}
\section*{\underline{June 7th}}

\prob{1}
We see two gradient flow lines on $S^1$ \TODO

Hence, the Morse DGA becomes
\[\F_2\{p_2\}\xrightarrow{\partial} \F_2\{p_1\}\rightarrow 0\]
where $\partial p_2 = 2p_1=0$.
\prob{2}

\prob{3}

\prob{4}

\prob{5}

\prob{6}

\prob{7}

\end{document}
