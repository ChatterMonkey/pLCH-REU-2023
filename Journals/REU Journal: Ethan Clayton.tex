\documentclass[11pt,oneside]{amsart}
\usepackage{alttpreamble}
% \usepackage{tocloft}
% \renewcommand{\cftsecleader}{\cftdotfill{\cftdotsep}}
\begin{document}

\author{Ethan Clayton}

\address{University of Illinois, Urbana-Champaign}
\email{ewc3@illinois.edu}

\title{REU Journal Entries: Ethan Clayton}

\maketitle

\tableofcontents
\newpage

\section{Journal Entry: June 22nd}

Over the weekend, we developed the algorithm to assign heights to each of the Reeb chords in the knot. Following this discovery, I spend the rest of the weekend computing the algorithms on a series of examples. This was both to gain practice with using the algorithm and an attempt to find an knot or particular representation of the knot where the algorithm would fail. I also worked a little with Maya and Fredrick in trying to prove properties about the algorithm, like the fact that it always finishes for knots in plat position and how it is affected by various Reidemiester moves. This algorithm is useful since it can give a 'canonical' assignment of the Reeb height and it gives only a small amount of distinct heights, which streamlines the computation for the Persistent Legendrian Contact Homology.
\newline

One possible way that the algorithm would fail on a knot, is if, in the set of remaining equations, no variable is always positive. In other words after a step is completed, every variable is negative in at least one remaining equation. As of yet we have not found this configuration in any knot diagram.
\newline

I also attempted to proved that the functioning of the algorithm is invariant under Reidemeister moves, meaning that if the algorithm finishes for a knot, it will also finish for that knot following a Reidemeister move. This combined with the fact that the algorithm always finishes for a knot in plat position, would have been enough to show that the algorithm always finishes. However, we were unable to prove this for Reidemester move 3. This is because the algorithm works on a global (i.e. the entire knot), rather than a local (one portion of the knot) scale. This means, the consequences of a Reidemester move on one park of the knot effect how the algorithm classifies generators in all parts of the knot, in a manner that my group and I do not entirely understand at this point. 
\newline

My current focus is to compute PLCH for many different knots in plat position. We will also compute the PLCH of these same knots using different assignments of Reeb heights. From this we can better understand how different Reeb heights affect how the PLCH picture looks and hopefully develop a notion of equivalence between Reeb height orderings. My group mates are also working on some code that will help to automate this proceeds so more examples can be computed faster. This code will also be useful in the future if we are working on large or very complex knots that are too cumbersome to compute by hand. We have already coded a way to solve the algroithm for knots in plat position.


\section{Journal Entry: June 27th}

Since the last journal entry, our group made a lot of good progress doing examples. We calculated the persistence homology for around 20 different knots. The algorithm we developed, along with the mathematical notebook were very helpful in speeding up out calculations and allowing us to compute more examples quicker. Maya also developed a python script to automatically compute the algorithms height assignments for knots in plat position. 
\newline

We then shifted out focus to calculating knots with height assignments different from those given from out algorithm, since these corespond to planar isotopy of the knot. We found that if we changed the height assignments, the starting and ending points of certain bars in the persistence homology could change but no bars, finite or infinite could be created or destroyed. Furthermore, finite bars could not turn into infinite bars and vice versa. We also have the outline of a formal proof for this.
\newline

We then looked at Reidemeister moves, specifically in the front projection. This is because Andy Legendrian isotopy of knots can be achieved by a series of front projection Reidemeister moves and planar isotopy. So if we understand all of the Reidemester moves and planar isotopy with respect to the persistence module structure we can create an equivalence relation of the barcodes.
\newline

After doing examples with Reidemeister move one, we found that it adds one par to the persistence structure. Specifically it is a finite bar in $H_0$ and that bar is generated by a single Reeb chord. We are still working on Reidmesiter moves two and three in the front projection, but we have some ideas. Reidemeister move two can be broken up into two cases, namely when the line crosses a left cusp and when the line crosses a right cusp. When the line crosses a left cusps we believe that it will have a similar effect as Reidemeister move one.
\newline

We have some insight into hove these moves effect the DGA and thus the persistence module. We believe these moves will correspond in some way to stable tame isomorphisms of the DGA, but we still need to understand which corresponds to stable moves and which correspond to tame moves and how they stable and tame things interact with the persistence modules.


\end{document}