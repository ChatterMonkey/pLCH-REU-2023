\documentclass[11pt,oneside]{amsart}
\usepackage{alttpreamble}
% \usepackage{tocloft}
% \renewcommand{\cftsecleader}{\cftdotfill{\cftdotsep}}
\begin{document}

\author{Ethan Clayton}

\address{University of Illinois, Urbana-Champaign}
\email{ewc3@illinois.edu}

\title{REU Journal Entries: Ethan Clayton}

\maketitle

\tableofcontents
\newpage

\section{Journal Entry: June 22nd}

Over the weekend, we developed the algorithm to assign heights to each of the Reeb chords in the knot. Following this discovery, I spend the rest of the weekend computing the algorithms on a series of examples. This was both to gain practice with using the algorithm and an attempt to find an knot or particular representation of the knot where the algorithm would fail. I also worked a little with Maya and Fredrick in trying to prove properties about the algorithm, like the fact that it always finishes for knots in plat position and how it is affected by various Reidemiester moves. This algorithm is useful since it can give a 'canonical' assignment of the Reeb height and it gives only a small amount of distinct heights, which streamlines the computation for the Persistent Legendrian Contact Homology.
\newline

One possible way that the algorithm would fail on a knot, is if, in the set of remaining equations, no variable is always positive. In other words after a step is completed, every variable is negative in at least one remaining equation. As of yet we have not found this configuration in any knot diagram.
\newline

I also attempted to proved that the functioning of the algorithm is invariant under Reidemeister moves, meaning that if the algorithm finishes for a knot, it will also finish for that knot following a Reidemeister move. This combined with the fact that the algorithm always finishes for a knot in plat position, would have been enough to show that the algorithm always finishes. However, we were unable to prove this for Reidemester move 3. This is because the algorithm works on a global (i.e. the entire knot), rather than a local (one portion of the knot) scale. This means, the consequences of a Reidemester move on one park of the knot effect how the algorithm classifies generators in all parts of the knot, in a manner that my group and I do not entirely understand at this point. 
\newline

My current focus is to compute PLCH for many different knots in plat position. We will also compute the PLCH of these same knots using different assignments of Reeb heights. From this we can better understand how different Reeb heights affect how the PLCH picture looks and hopefully develop a notion of equivalence between Reeb height orderings. My group mates are also working on some code that will help to automate this proceeds so more examples can be computed faster. This code will also be useful in the future if we are working on large or very complex knots that are too cumbersome to compute by hand. We have already coded a way to solve the algroithm for knots in plat position.


\section{Journal Entry: June 27th}

Since the last journal entry, our group made a lot of good progress doing examples. We calculated the persistence homology for around 20 different knots. The algorithm we developed, along with the mathematical notebook were very helpful in speeding up out calculations and allowing us to compute more examples quicker. Maya also developed a python script to automatically compute the algorithms height assignments for knots in plat position. 
\newline

We then shifted out focus to calculating knots with height assignments different from those given from out algorithm, since these corespond to planar isotopy of the knot. We found that if we changed the height assignments, the starting and ending points of certain bars in the persistence homology could change but no bars, finite or infinite could be created or destroyed. Furthermore, finite bars could not turn into infinite bars and vice versa. We also have the outline of a formal proof for this.
\newline

We then looked at Reidemeister moves, specifically in the front projection. This is because Andy Legendrian isotopy of knots can be achieved by a series of front projection Reidemeister moves and planar isotopy. So if we understand all of the Reidemester moves and planar isotopy with respect to the persistence module structure we can create an equivalence relation of the barcodes.
\newline

After doing examples with Reidemeister move one, we found that it adds one par to the persistence structure. Specifically it is a finite bar in $H_0$ and that bar is generated by a single Reeb chord. We are still working on Reidmesiter moves two and three in the front projection, but we have some ideas. Reidemeister move two can be broken up into two cases, namely when the line crosses a left cusp and when the line crosses a right cusp. When the line crosses a left cusps we believe that it will have a similar effect as Reidemeister move one.
\newline

We have some insight into hove these moves effect the DGA and thus the persistence module. We believe these moves will correspond in some way to stable tame isomorphisms of the DGA, but we still need to understand which corresponds to stable moves and which correspond to tame moves and how they stable and tame things interact with the persistence modules.

\section{Journal Entry: July 6th}

After trying to work on determining some invariant structure in the finite bars of persistence homology, both looking at singularly generated bars, and later two different definitions of $H$-contractible bars, we have decided to focus on a simple case, with added constraints on the heights.
\newline

We added constraints to the Reeb heights that they all must be strictly positive integers, and that the total unsigned area of the knot must be a fixed value. We hope that this we restrict the ways in which the finite bars can change and give us some insights as th what to do in more general cases.
\newline

We used some integer and linear programming tool build into mathematica to calculate the minimum possible unsigned area that a knot could have with out added restriction. This also gave us an explicit assignment of Reeb heights that satisfied that minimal area. We also found another function which could give us all possible Reeb height assignments for a specified total unsigned area.
\newline

We did an example on a knot in plat position, which could have both Reidemiester moves 2 and 3 done two it. We calculated the minimal area to be 20, and after performing Reidemeister move 3, the new minimal area is 27.

\section{Journal Entry: July 10th}

I was working on some examples of height assignments and bar lengths involving the trefoil. I restricted the heights to be positive integers and the area of each loop to have a strictly positive area. Under these rules, the original trefoil had a finite length bar, whose length could be made no less than there. After performing Reidemeister move two's i found that they added bars whose lengths could be made 1, the minimum possible length of a bar. I hoped to see that the number of bars whose length was not 1 could be an invariant in the same vein as H-contractible. However after performing 2 Reidemeister move 2's and the one Reidemeister move 3 on the trefoil. I found that there were two distinct bars, that both could not be made to have a height of 1.
I had a bar born by $b_1$ and killed by $b_2$, and a second bar born by $b_1 + a_3 + c_1$ and killed by $a_1$, and I had the following inequalities give by area loops
$b_1 \geq (a_4 + a_3 + 1) \geq b_2 + 2$, and $c_1 \geq (b_1 + a_3 + 1) \geq a_1$

This knot is also caused the algorithm we use to assign heights (aka the flooding) to fail to finish, getting stuck after filling in $a_1, a_2, a_3$

\section{Journal Entry: July 11th}

I tried to come up with some canonical height assignments that would behave nicely under the Reidemeister moves. First I tried to set all of the two-gons and the triangles to have area zero, and minimize the total area. This meant that any new bar created would have exactly 0 length, which was pretty promising. This worked in some cases, but sometimes there were no solutions where the Reeb heights were all positive. So I tried to set a few of the heights to 0 in order to make all of the triangles work out. This was possible, however, after performing two Reidemeister move two's and then a Reidemeister move three on the trefoil, I made the entire top of the knot into triangles, which made the original finite bar which did not initially have zero length have zero length, so this forced me to rethink. While I was thinking, Maya had  some geometric insight into which bars could be ignored by seeing if the disks that created them could be 'pulled out' of the knot, or if they were trapped so I went to work on that idea.

Maya's Idea was to look at the disk that corresponded to the finite bar and see if it was linked to the rest of the knot or see if it could  be removed. The idea was that bars which had disks that could be removed we inessential and could be crossed out, where bars that had linked disks were essential. We did some examples and the idea seemed to work, giving the same number of linked bars for knots we knew were planar isotopy. This worked under planar isotopy and Reidemeister move two, but the argument failed for Reidemeister move three since previously linking strands could be moves out through a crossing in the disk. We then tried to refine our definition of linking disks so that it would work for all of the Reidemeister moves since we thought that the idea was very promising, but we were unable to come up with a definition that worked

\end{document}