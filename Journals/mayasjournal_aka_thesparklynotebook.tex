\documentclass[11pt,oneside]{amsart}
\usepackage{alttpreamble}
% \usepackage{tocloft}
% \renewcommand{\cftsecleader}{\cftdotfill{\cftdotsep}}
\begin{document}
\author{Maya Basu}


\address{University of California, Berkeley}
\email{mab@berkeley.edu}



\title{Journal of the Resident Sparkle Squid}
\maketitle



\section{1st Entry}

\subsection{Saturday} The main question is can we come up with an algorithm to assign heights? One homework question was to do this for an example, and to make the inequalities work trivially, I jokingly set one of the Reeb heights to $10^8$. This is what was my eventual inspiration for how we could assign heights. My idea was that there were certain chords which could be made arbitrarily big which allowed other choords to be anything, and we would still get positive area loops. I realized this corresponded to, geometrically, stretching out the outside loops so that generators in that area were "solved" automatically. 

Then I had the next realization - by "solving" loops, I freed them up to "solve" more loops, and so I could propagate through the knot. 
We spent the rest of the day refining our understanding of exactly how the algorithm would work. 



\subsection{Monday} Maybe the algorithm stops and doesn't end up propagating through the entire knot - in this case we shouldn't get a solution for every knot. I spent a lot of time trying to rearange knots - putting the interesections in some sort of grid and drawing the knot lines between them - to get some way to prove that the algorithm would finish. My hypothesis was that since the algorithm shouldn't necessarily work for every set of inequalities, it must be some geometric feature of the knot. Essentially, I figured because it was one long strand, this must make some restictions of the inequalities present which would help solve the problem.

We found some interesting relations based on the fact that only certain numbers of + or - generators can show up in teh inequalities, however, these wer ultimitly unhelpful.

Finally, I realized that what I was trying to do - rearrange a knot in a controlled manner was actually already a thing! I realized I could use plat position. From here, it seems very possible to complete the proof, and I eventually figured out one way to do it.


\subsection{Tuesday/Wednesday}

The biggest question next seemed to be the issue of contractible chords. As Ethan pointed out, all of the chords in the extras category were contractible, however, we found that intruducing a stabilization gave us a contractible chord in the second class. At first I was confused about why this loop would be contractible when the side loops in the trefoil were not, but then thinking about the issue algebraically, and clarifying a possible definition of contractibility, which I added to the main log, it became imediatly obvious why this was true. 

This is also when we realized the difference between contractible and super-contractible. The discussion can up while trying to go from the categories given by the algorithm to an actual height assignment. The plan was to give all contractible chords of height one, but this made very clear the issue that some contractible chords required modifying others to let them shrink - only the extras, the super contractible chords, could all necessarily be made arbitrarily small. 

Wednesday evening I started on python code to go from the plat position representation of a knot to height assignments using what we had decided to call the flooding algorith.


\subsection{Thursday}

Finished the code! This took most of the day. 

Additionally, we discussed out next steps. It seems like to start generating visuals we should try to compute the barcode of many knots. Only finding the homology would leave out information about the basis of generators, but if we found the generators at each step of the filteration, we could easily recover the homology afterwards.


\subsection{Friday}

Wrote code that identifies all cycles up to a certain height, given the height filteration and the differentials. Hopefully this will help us create visuals and understand persistance better


\section{2nd Entry}

\subsection{Saturday}

Today was mostly grinding examples, all three of us tried varying height assignments(basically the effect of a planar isotopy) and found that this just moved bars but didn't actually change the number of bars or their status as finite/infinite. 

Additionally, I started code that computes the persistant homology given the differentials, the grading, and the height assignments. While this will pribably be useful to help us check our work, my end goal is to have code that can automatically compute the barcode of any given knot, given the differentials, grading, and height assignments. This requires tracking the generators, however I have a plan of how to do this. Because I already wrote something that more or less looks for cycles, I can usilize this at each step of the persistance. That I know the persistant homology (I wil be calculating the kernel/image of each map between gradings) will tell me how many things to look for, which I think will make my task easier. For example, if I know that the kernel of a certain map from grading 1 to 0 has a certain dimentional kernel, then I know how many independent cycles to look for. I shouldn't technically need to compute the homology first, but this gives me a sanity check, and a way to write the code more iterativly, which I think I need since this is a fairly complex procedure.





\subsection{Sunday}

I suppose that this belongs is Sunday, since I finished at about 12:30 am, but I did the majority of the code for the homolgy computitions, as at 10:30 pm about, my brain decided that it was inspired, and that I absolutly couldnt call asleep untill I coded out my ideas.  ... lol.  Anyway, today was basically the same as Saturday with all of us doing examples, except Ethan worked out barcode changes for the first riedemister move, and both of them pointed out I needed to be doing computations in mod 2 for the homology code. Something isn't working quite right, but this isn't our first priority, so I will stop working on this for now. Additionally, Ethan and I spent a while trying some simple knots trying to get move three to work, but the results couldn't me augmented. 

\subsection{Monday}

Our focus today was reidemister moves 2 and three. We started out optimisitally trying these moves on the checanov to see what happened but we ran into issues. First, It does not appear that you can even do reidemister move three on the chechanov. I spent a while trying to come up with the plat positions of various knots that might host move three, but I couldn't find a single one that was also augmentable. This is clearly going to be an issue. Fredrick was trying to do reidemister move two and eventually we all ended up working on the example. It seemed like we got the promising result that only a single finite bar was added to $H_{-1}$, but then when computing the last step in persistance, the dimentions stopped working out. Something was wrong and after trying to figure out what for a while we gave up. 

After taking a break, I realized that we shouldn't actually have to try examples, or even find examples to figure out what happened to the bars. If we could find how the DGA transformed directly then possibly we could infer changes to the barcode directly from this. I tried this with the three point move and found a morphism of the differentials that corresponded. It seemed like none of the disks changes except in very controlled ways. THis was great! Then all three of use came up with an argument for a left reidemister two move (a left cusp moving through a vertical strand) which gives us confirmation of a very simple transformation of barcodes. However, returning back to move 3, I couldnt find a way to relate the transformation of the differentials to a transformation of the bars of the barcode. This will be a goal for tommorow. 


\subsection{Wednesday}

Previously we were hard pressed to find a simple and valid knot which we could preform reidemister move three on, however Ethan realized that in plat position, actually we can do move three on the chekanov knot. We we calculated the persistant homology before and after doing this move. The results were not good, so much changed it was very difficult to identify how things have changed. This was very disheartening since it gave a counter example to many of the things we had guess might possibly be invariant. 

However, this gave me an idea. It seemed like the issue here was the choice of height assignment. The longer I thought about it the more convinced I was that I should try to find a height assignment that was invariant under moves three and two. This way, after a move three was done, the height assignments wouldn't change, and any change in the barcode would be due to the actual geometry of the reidemister move alone. The problem at the moment was that after the reidemister move, we had reassigned heights, and now that the generators came in a different ordwer in plat position, a bunch of the area loops, and thus the inequalites, and thus the height assignments were changed. 

So, what I wanted was an assigment of heights that didn't change the area inequalities after the move was done. By considering all possible area loops contacting a three point move, 



\subsection{Thursday}
\end{document}