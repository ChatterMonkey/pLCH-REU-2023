\documentclass[11pt]{amsart}
\usepackage{alttpreamble}
% \usepackage{tocloft}
% \renewcommand{\cftsecleader}{\cftdotfill{\cftdotsep}}
\begin{document}
\author{Maya Basu}
\author{Ethan Clayton}
\author{Fredrick Mooers}

\address{University of California, Berkeley}
\email{mab@berkeley.edu}

\address{University of Illinois, Urbana-Champaign}
\email{ewc3@illinois.edu}

\address{Virginia Tech}
\email{mooersfl24@vt.edu}

\title{Persistent Legendrian contact homology in $\R^3$}



\subjclass{53D42; 53D10, 57K10; 55N31}
\keywords{Legendrian contact homology, Legendrian knot, Persistent Homology}

\maketitle

\tableofcontents
\newpage

\section{June 17th: Proposed Algorithm}
Goal: Find an algorithm for determining heights

Inspiration: Think of the knot in terms of small tiles - each smallest area is a subsection. We can iteratively move through the knot taking care of one of these tiles at a time. My idea is that outside loops are easier to solve and then we can narrow down the number of relevant generators.

Description:

First find the first free generators, these are the generators which are crossings of loops that are completely outside the knot, or generators which always show up positive in every set of inequalities obtained by the area calculation of each tile. We can show that any exterior loop gives a maximally free generator because going around the loop part, or the other side of the crossing gives a positive value (ccw). This means that this loop can be scaled to be as big as necessary to make areas positive.

\includegraphics[width=4in,angle=-90]{Exterior Loops.JPG}


Now, we shade out the area that is contained on the other side of the loop, the other area whose area inequality includes a positive value from this first free generator. Algebraically, this means that we cross out the equations which contain this first free generator. Repeat this for all free generators.

Now, out of the uncrossed out equations, those generators which appear only as positive are the second free generators. This is the set of generators which have, with respect to ccw loops through the small area tiles, only positive values, if we discount any negative values in already crossed out loops. 

\includegraphics[width=4in]{Visualization of Algorithm.JPG}

In this manner, continue through to the nth free generators, obtaining n classes of freedom, "the extras" are class n, which are never only positive in the remaining equations
\[f^{1}_1 \cdots f^{1}_{k_1} > f^{2}_1 \cdots f^{2}_{k_2}> \cdots > f^{n-1}_1 \cdots f^{n-1}_{k_{n-1}} > k_n \text{ extras }\]

Now we work backwards - assigning those in the extras class (these are all contractible, though not all Reeb chords have to be nth free) to length 1, and then those in the $n-1$th class have length $k_n \cdot 1 + 1$ (+1 is so that we don't get exact equality) and those in the $n-2$nd class have length $ (k_n \cdot 1 + 1)\cdot k_{n-1} + 1$ and those in the $n-3$rd class $ ((k_n \cdot 1 + 1)\cdot k_{n-1} + 1)\cdot k_{n-2} + 1$ etc...




Hypothesised equivalency for height lists of the same number of generators

Perhaps an appropriate equivalency relation while include the following notion of equivalence:

Suppose that we have two classing systems of generators. Each produces many strict orderings of generators which still respect this class system. If the intersection of the set of strict orderings produced by two classing systems is nonzero, perhaps we should we consider these equivalent.




Further note:

These classes give us a filtration which is an action filtration with equities set at every opportunity. 

Examples:

add examples

Summary and further questions:

If this algorithm ends (i.e covers the entire knot) we have a way to assign heights to all of the generators. This is not unique, and is very "inefficient" in packing the values of generators closely. For example, setting those in the nth class equal to 1, we would perhaps want those in the 1st class to be as small as possible. We can do this by considering the loops one at a time backwards, setting the positive values as small as possible so that the loops are barely positive. However, if there are say two positive junctions in one of these nth loops, we no longer have uniquely determined values. 

The biggest question here is does the algorithm finish - will all areas of any loop become shaded after finite iterations (finite is guaranteed as long as the number of generators covered always increases by at least one for each class)

Additionally question, where to contractible chords show up? We know that all chords in the nth class are contractible because they are not required to be greater than any other value. However, doing the 1st Reidemiester move in the front projection and then going to the the Lagrangian projection gives us a crossing which, if the the Reidemiester moves was done along the top of a loop belonging to class 1, then now is in the second freedom class. 


\includegraphics[]{General Information/path184.png}



\section{June 19th: Proof of algorithm in a special case}

Hand wavy thought process and motivation: 

The main issue is what happens if the algorithm doesn't finish - meaning that not all equations get crossed out. This would happen if we have inequalities involving several variables where each variable appears as both negative and positive in these inequalities. Can we come up with a counter example? It seems like trying to come up with a knot satisfying a set of inequalities with this contradiction is very difficult, if not impossible. In trying to generate valid knots I realized the difficulty was in the orientation of the crossings - because the loops can only be sideways in the Lagrangian projection (or else we have just one negative crossing in a loop). 

In trying to come up with a rigid way to quantize the knot to a grid of sorts, I realized that this already exists - plat position.

So assume that we take a knot in plat position and apply Ng's algorithm to get a knot in the lagrangian projection. Now we have something that is extremely tame and can prove that the algorithm finishes.



Lemma: The algorithm terminates and gives an ordering of heights witch satisfy the area inequalities when applied to any lagrangian projection (via Ng's algorithm) of the plat position of the front projection of a legendrian knot. 

Proof:

Consider a knot in plat position which has two types of crossings. Considering a row (we call a region such as the one shaded below a "row", a strip where it ends in a loop)



there are crossings within rows and crossings between rows. Working our way from the right






Further observations:

All extras are "super" contractable (these are generators which are never responsible for shading a loop - therefore they can \bold{all} be set equal to zero \bold{simultaniously}.

Not all contractible chords are super contractible - some chords may go to zero iff one of the other positive generators in their loop increases by some finite amount. 


If there are no extras (super contractible chords) than the last class are contractible, as the lack of extras implies the last set of equations to be crossed out contains no negatives. 


Contractible chords can be in classes other than the last/extras class. For example, a stabilization introduced a contractible chord in the second class.









\section{June 21st}


Algebraic description of contractability:

A contractible crossing is a generator $a_i$ in the area inequalities where for each inequality that $a_i$ appears as positive, there are no other negative terms, or if there is one or more negative terms, there is at least one other positive term.

This is a necessary condition for contractibility, but not sufficient, as if we take the "actual" definition of contractibility to be a legandrian isotopy up to the endpoint where the height goes to zero. This can not happen if the above requirement is not satisfied, however, the above requirement does not automatically guarantee the existence of such a legendrian isotopy. 


Stabilization:

Adding a stabilization with generator $a_j$ creates a contractible reed chord, because we add two things to the area inequalities: $a_j>0$ and we add a positive $a_j$ to one other area inequality. Because we previously had a valid knot, there must have been at least one other positive in this latter equation, so the added $a_j$ satisfies the necessary conditions of contractibility. 




Further Questions:



Can we list all of the possible orderings of Reeb heights?
Can we do a bunch of examples of computing FLCH?
When we have classes with multiple elements we can sometimes swap the orders. If we take the Reeb heights after swapping, how does this change FLCH (examples)?
Are finite intervals invariant under legandrian isotopy?

Double shading corresponds to contractible, extras are stupid contractible. (proven)


Removing a crossing and rerunning the algorithm - if the chord is contractible it should still finish?

Could we prove that the algorithm finishing is preserved through isotopy by considering the equations under stable tame isomophism?

Could we use grid projection to prove things?

Make a python script to compute height assignments. (completed


Is non-switchability a property of only having two classes.


We should call this the flooding algorithm - knot is flooded  - each class is a "tier" 
\newline

When drawing the bar-code, which generator or sums of generators should we use? In some cases it seems that it is ambiguous and we have a choice of which generators to use, and each choice leads to different effects when the heights of certain generators are changed.

Hypothesis, when changing the height of one generator, it will affect the bar-code only by moving the birth and death time of certain lines, rather than by introducing additional lines.

How will the bar-code be affected by Reidemeister moves (i.e. under the addition of new generators)?

Work flow:



Draw knot in plat position, get generators

Ask for z-graded differential

Compute height assignments with python

















\end{document}